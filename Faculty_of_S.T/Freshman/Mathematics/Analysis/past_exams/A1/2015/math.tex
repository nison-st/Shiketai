\documentclass{jarticle}
\begin{document}
以下の設問1から5に答えよ.解答は 解答用紙の所定の欄に記入すること.
\begin{enumerate}
    \item \[\lim_{x\to 0}\frac{\frac{\cos x}{1+x^{2}}+a+bx^{2}}{x^{4}}\]
        が有限の極限値をもつように
	    定数a,bを定め,そのときの極限値を求めよ.
    \item 
        \(\sin(x-y)-(x+y)\cos(x-y)\)の(0,0)におけるテイラー展開において,\(y^{3}\)の項および\(x^{5}\)の項を決定せよ.
    \item 
    \begin{enumerate}
        \item [(1)]\(f(x,y)=\frac{1}{\pi}(x^{2}-3xy)-x+2y+\sin{x}-\cos{y}-(\frac{5}{3}\pi+\frac{\sqrt{3}}{2})=0
	      	\)により定まる陰関数\(y=\varphi(x)\)で
	      	\(x=0\)のとき\(y=\frac{5}{6}\pi\)を満たすものがただ一つ存在することを示し,\(\frac{d\varphi}{dx}(0)\)を求めよ.
        \item [(2)]さらに\(\frac{d^{2}\varphi}{dx^{2}}(0)\)を求めよ.
    \end{enumerate}
	      
    \item 
    2変数関数\(g(x,y)=3x^{2}y+y^{3}-12x^{2}-75y\)を考える.
    \begin{enumerate} 
        \item [(1)]\(g(x,y)\)の停留点をすべて求めよ.
        \item [(2)](1)で求めた停留点の各点について,極大点,極小点,鞍点,あるいはいずれでもないか,を判定せよ.
    \end{enumerate}
    \item \(\varphi(x,y)=6x^{4}+x^{2}+y^{2}-1=0\)を満たしながら(x,y)が動くとき,\(f(x,y)=x^{2}+y^{2}\)
    の最大値,最小値とそれらを与える(x,y)を\underline{ラグランジュの乗数法を用いて}すべて求めよ.
	\end{enumerate}
\end{document}
