\documentclass[a4paper,10pt,report]{amsart}
    \usepackage{bm,amsmath,amssymb,framed,cases,euscript,cleveref,newtxtext,newtxmath}
    \usepackage{remreset}
    \usepackage[dvipdfmx]{graphicx}
    \makeatletter
        \renewcommand{\theequation}{% 式番号の付け方
        \thesection.\arabic{equation}}
        \@addtoreset{equation}{section}
    
        \renewcommand{\thefigure}{% 図番号の付け方
        \thesection.\arabic{figure}}
        \@addtoreset{figure}{section}
        \renewcommand{\thetable}{% 表番号の付け方
        \thesection.\arabic{table}}
        \@addtoreset{table}{section}
    \makeatother
    %
    \theoremstyle{plain}
    %
    \theoremstyle{definition}
    \newtheorem{defn}{Def.}[section]
    \newtheorem{thm}{Thm.}[section]
    \newtheorem{lem}{Lem.}[section]
    \newtheorem{cor}{Cor.}[section]
    \newtheorem{prop}{Prop.}[section]
    \newtheorem{law}{Law.}
    \newtheorem{req}{Req.}[section]
    \crefname{req}{Requirement}{Requirements}
    %
    \theoremstyle{remark}
    \newtheorem{rem}{Rem}
    \newtheorem{prf}{Prf.}
%
    \let\normalref\ref
\renewcommand{\ref}{\cref}
    \newcommand{\Tabref}[1]{表\ref{#1}}
    \newcommand{\Equref}[1]{式 (\ref{#1})}
    \newcommand{\Figref}[1]{図\ref{#1}}
    %
    %
    \title{Answer for 1B 2016}
    \author{Naoki Yano}
    \date{\today}
    %
\begin{document}
    \maketitle
    \begin{enumerate}
        \item 
        \begin{equation*}
            \frac{x^{3}-3}{x^{3}-x^{2}-x+1}=\frac{x^{3}-x^{2}-x+1+(x^{2}+x-4)}{x^{3}-x^{2}-x+1}
        \end{equation*}
        \begin{equation*}
            =1+\frac{x^{2}+x-4}{x^{3}-x^{2}-x+1}
        \end{equation*}
        ここで
        \begin{equation*}
            x^{3}-x^{2}-x+1=(x-1)^{2}(x+1)
        \end{equation*}
        だから
        \begin{equation*}
            \frac{x^{2}+x-4}{x^{3}-x^{2}-x+1}=\frac{A}{{(x-1)}^{2}}+\frac{B}{x-1}+\frac{C}{x+1}
        \end{equation*}
        とおけて, 
        \begin{equation*}
            x^{2}+x-4=A(x+1)+B(x+1)(x-1)+C{(x-1)}^{2}
        \end{equation*}
        \begin{equation*}
            x^{2}+x-4=(B+C)x^{2}+(A-2B)x+A-B+C
        \end{equation*}
        係数を比較して,
        \begin{equation*}
            \begin{cases}
                B+C=1\\
                A-2B=1\\
                A-B-C=-4
            \end{cases}
        \end{equation*}
        これを拡大係数行列にして連立方程式を解くと, 
        \begin{equation*}
            \begin{bmatrix}
                0 & 1 & 1 & 1\\
                1 & -2 & 0 & 1\\
                1 & -1 & -1 & -4
            \end{bmatrix}
            \to \begin{bmatrix}
                1 & -1 & -1 & -4\\
                0 & 1 & 1 & 1\\
                0 & 0 & 1 & 3
            \end{bmatrix}
        \end{equation*}
        \begin{equation*}
            \therefore C=3,B=-2,A=-3
        \end{equation*}
        \begin{equation*}
            I=\int1dx+\int\left(\frac{-3}{{(x-1)}^{2}}+\frac{-2}{x-1}+\frac{1}{x+1}\right)dx
        \end{equation*}
        \begin{equation*}
            =x+\frac{3}{x-1}+\log\left|\frac{x+1}{{(x-1)}^{2}}\right|+\mathrm{const.}
        \end{equation*}
        \item
        \begin{equation*}
            \mathcal{D}:=\{(x,y)\mid0\leq y\leq 1,y^{3}\leq x\leq 2-y^{2}\}
        \end{equation*}
        これをグラフに書くと
        \begin{equation*}
            \mathcal{D}'=\left \{(x,y)\middle|0\leq x\leq 2,
            \begin{cases}
                0\leq y\leq \sqrt[3]{x}&(0\leq x\leq 1) \\
                0\leq y\leq \leq \sqrt{2-x}&(1\leq x\leq 2)
            \end{cases}
            \right \}
        \end{equation*}
        從って定積分は
        \begin{equation*}
            I=\int_{0}^{1}\left(\int_{0}^{\sqrt[3]{x}}\frac{y^{2}}{x\sqrt{2-x}}dy\right)dx+\int_{1}^{2}\left(\int_{0}^{\sqrt{2-x}}\frac{y^{2}}{x\sqrt{2-x}}dy\right)dx
        \end{equation*}
        \begin{equation*}
            =\int_{0}^{1}\frac{1}{3\sqrt{2-x}}dx+\int_{1}^{2}\frac{2-x}{3x}dx={\left[-\frac{2}{3}\sqrt{2-x}\right]}_{0}^{1}+\left[\frac{2}{3}\log{}x\right]_{1}^{2}-\left[\frac{1}{3}x\right]_{1}^{2}
        \end{equation*}
        \begin{equation*}
            =\frac{2}{3}(\sqrt{2}+\log{2})-1
        \end{equation*}
        \item 
    \end{enumerate}
    %
\end{document}