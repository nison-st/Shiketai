\documentclass[a4paper,10pt,report]{amsart}
    \usepackage{bm,amsmath,amssymb,framed,cases,euscript,cleveref,newtxtext,newtxmath}
    \usepackage{remreset}
    \usepackage[dvipdfmx]{graphicx}
    \makeatletter
        \renewcommand{\theequation}{% 式番号の付け方
        \thesection.\arabic{equation}}
        \@addtoreset{equation}{section}
    
        \renewcommand{\thefigure}{% 図番号の付け方
        \thesection.\arabic{figure}}
        \@addtoreset{figure}{section}
        \renewcommand{\thetable}{% 表番号の付け方
        \thesection.\arabic{table}}
        \@addtoreset{table}{section}
    \makeatother
    %
    \theoremstyle{plain}
    %
    \theoremstyle{definition}
    \newtheorem{defn}{Def.}[section]
    \newtheorem{thm}{Thm.}[section]
    \newtheorem{lem}{Lem.}[section]
    \newtheorem{cor}{Cor.}[section]
    \newtheorem{prop}{Prop.}[section]
    \newtheorem{law}{Law.}
    \newtheorem{req}{Req.}[section]
    \crefname{req}{Requirement}{Requirements}
    %
    \theoremstyle{remark}
    \newtheorem{rem}{Rem}
    \newtheorem{prf}{Prf.}
%
    \let\normalref\ref
\renewcommand{\ref}{\cref}
    \newcommand{\Tabref}[1]{表\ref{#1}}
    \newcommand{\Equref}[1]{式 (\ref{#1})}
    \newcommand{\Figref}[1]{図\ref{#1}}
    %
    %
    \title{Analysis}
    \author{Naoki Yano}
    \date{\today}
    %
    \begin{document}
    \maketitle
    \tableofcontents
    \part{実数と連続}
    \section{実数の公理}
    \begin{leftbar}
        \begin{defn}加群\par
            \begin{enumerate}
                \item 交換律
                \begin{equation*}
                    a+b=b+a.
                \end{equation*}
                \item 結合律
                \begin{equation}
                    (a+b)+c=a+(b+c)
                \end{equation}
                \item 零元の存在
                \begin{equation}
                    \exists0\in\mathbb{R},\forall a\in\mathbb{R},a+0=a.
                \end{equation}
                \item 逆元の存在
                \begin{equation}
                    \forall a\in\mathbb{R},\exists -a\in\mathbb{R},a+(-a)=0
                \end{equation}
            \end{enumerate}
        \end{defn}
    \end{leftbar}
    \part{微分法}
    \part{積分法}
    \section{逆三角関数}
    \begin{leftbar}
        \begin{defn}逆三角関数\par
            \begin{equation}
                x=\sin{y},-\frac{\pi}{2}\leq y\leq\frac{\pi}{2}
            \end{equation}
            には逆関数が存在して, これを
            \begin{equation}
                y=\mathrm{Sin}^{-1}x,-1\leq x\leq 1
            \end{equation}
            と書く. また
            \begin{equation}
                x=\cos{y},0\leq y\leq \pi
            \end{equation}
            にも逆関数が存在して,
            \begin{equation}
                y=\mathrm{Cos}^{-1}x,-1\leq x\leq 1
            \end{equation}
            と書く. 
        \end{defn}
    \end{leftbar}
    \begin{leftbar}
        \begin{prop}逆三角関数の表示
            \begin{equation}
                \int\frac{dx}{\sqrt{1-x^{2}}}=\mathrm{Sin}^{-1}x+\mathrm{const.}
            \end{equation}
            \begin{equation}
                \int\frac{dx}{\sqrt{1-x^{2}}}=-\mathrm{Cos}^{-1}x+\mathrm{const.}
            \end{equation}
        \end{prop}
    \end{leftbar}
    \begin{leftbar}
        \begin{defn}
            \begin{equation}
                x=\tan{y},-\frac{\pi}{2}<y<\frac{\pi}{2}
            \end{equation}
            \begin{equation}
                y=\mathrm{Tan}^{-1}x,-\infty<x<\infty
            \end{equation}
        \end{defn}
    \end{leftbar}
    \begin{leftbar}
        \begin{prop}
            \begin{equation}
                \int\frac{1}{1+x^{2}}=\mathrm{Tan}^{-1}x+\mathrm{const.}
            \end{equation}
        \end{prop}
    \end{leftbar}
    \section{有理関数}
    \begin{leftbar}
        \begin{defn}
            \begin{equation}
                R()
            \end{equation}
        \end{defn}
    \end{leftbar}
    \section{多変数関数の積分}
    \section{累次積分}
    \section{広義積分}
    \section{変数変換}
    \begin{leftbar}
        \begin{defn}ヤコビ行列式\par
            \(x=\varphi(u,v),y=\psi(u,v)\)と変数変換したときのヤコビ行列式は
            \begin{equation}
                J(u,v)=\frac{\partial(\varphi,\psi)}{\partial(u,v)}=\left|
                \begin{array}{cc}
                    \frac{\partial\varphi}{\partial u} & \frac{\partial\varphi}{\partial v}\\
                    \frac{\partial\psi}{\partial u} & \frac{\partial\psi}{\partial v}
                \end{array}
                \right|
            \end{equation}
        \end{defn}
    \end{leftbar}
    \begin{leftbar}
        \begin{defn}変数変換\par
            \(x=\varphi(u,v),y=\psi(u,v)\)と変数変換したとき
            \begin{equation*}
                \iint_{\mathcal{D}}f(x,y)dxdy=\iint_{\mathcal{D}'}f(\varphi(u,v),\psi(u,v))J(u,v)dudv
            \end{equation*}
        \end{defn}
    \end{leftbar}
    \section{体積}
    \section{質量と重心}
    \section{慣性モーメント}
    \section{曲面積}
    \begin{leftbar}
        \begin{thm}曲面積\par
            \(\mathcal{D}\subset\mathbb{R}^{2}\)から\(\mathbb{R}\)への写像を
            \begin{equation}
                f:\mathcal{D}\to \mathbb{R}
            \end{equation}
            として
            \begin{equation}
                \mathcal{A}=\left \{(x,y,z)\middle|(x,y)\in\mathcal{D},z=f(x,y)\right \}
            \end{equation}
            で表される曲面の面積は
            \begin{equation}
                S=\iint_{\mathcal{D}}\sqrt{1+f_{x}^{2}+f_{y}^{2}}dxdy
            \end{equation}
            で表される. 
        \end{thm}
    \end{leftbar}
    \begin{prf}
        
    \end{prf}
    \begin{leftbar}
        \begin{prop}曲面の法線ベクトル\par
            曲面
            \begin{equation*}
                \mathcal{D}:f(x,y,z)=\mathrm{const.}
            \end{equation*}
            上の点\((x,y,z)\)における法線ベクトルは
            \begin{equation}
                \nabla f(x,y,z)
            \end{equation}
            である.
        \end{prop}
    \end{leftbar}
    \section{ベクトル解析}
    \begin{leftbar}
        \begin{defn}スカラー場\par
            ベクトルからスカラーへの写像を
            \begin{equation}
                f:V\to K
            \end{equation}
            としたとき, 組み合わせ
            \begin{equation}
                (\bm{x},f(\bm{x}))
            \end{equation}
            をスカラー場という.
        \end{defn}
    \end{leftbar}
    \begin{leftbar}
        \begin{defn}ベクトル場
            ベクトルからベクトルへの写像を
            \begin{equation}
                f:V\to V
            \end{equation}
            としたとき, 組み合わせ
            \begin{equation}
                (\bm{x},f(\bm{x}))
            \end{equation}
            をベクトル場という. 
        \end{defn}
    \end{leftbar}
    \begin{leftbar}
        \begin{defn}微分演算子ベクトル (ナブラ)\par
            \begin{equation}
                \nabla :=\begin{bmatrix}
                    \frac{\partial}{\partial x}\\
                    \frac{\partial}{\partial y}\\
                    \frac{\partial}{\partial z}
                \end{bmatrix}
            \end{equation}
        \end{defn}
    \end{leftbar}
    \begin{leftbar}
        \begin{defn}勾配[gradient]\par
            \begin{equation}
                \nabla f
            \end{equation}
        \end{defn}
    \end{leftbar}
    \begin{leftbar}
        \begin{defn}発散\par
            \begin{equation}
                \nabla{}\cdot{}f
            \end{equation}
        \end{defn}
    \end{leftbar}
    \begin{leftbar}
        \begin{defn}回転\par
            \begin{equation}
                \nabla{}\times{}f
            \end{equation}
        \end{defn}
    \end{leftbar}

    \part{線績分・面積分}
    \begin{leftbar}
        \begin{defn}曲線\par
            \begin{equation*}
                \Gamma:=\{\bm(x)(t)\mid a\leq t\leq b\}
            \end{equation*}
            を曲線という. 
        \end{defn}
    \end{leftbar}
    \begin{leftbar}
        \begin{defn}線績分\par
            \begin{equation}
                \int_{\Gamma}fds:=\int f(\bm{x}(s))\sqrt{{\left(\frac{dx_{1}(s)}{ds}\right)}^{2}+{\left(\frac{dx_{2}(s)}{ds}\right)}^{2}+{\left(\frac{dx_{3}(s)}{ds}\right)}^{2}}ds
            \end{equation}
            を線績分という. 
        \end{defn}
    \end{leftbar}
    \begin{leftbar}
        \begin{defn}ベクトル場の線積分\par
            ベクトル場の線積分は
            \begin{equation}
                \int_{\Gamma}\bm{f}\cdot d\bm{S}=\int_{\Gamma}\bm{f}\cdot\bm{n}dS=\int \bm{f}\cdot\bm{n}\sqrt{{\left(\frac{dx_{1}(s)}{ds}\right)}^{2}+{\left(\frac{dx_{2}(s)}{ds}\right)}^{2}+{\left(\frac{dx_{3}(s)}{ds}\right)}^{2}}ds
            \end{equation}
        \end{defn}
    \end{leftbar}
    \begin{leftbar}
        \begin{defn}面積分\par
            \(z=\varphi(x,y)\)で表される曲面における\(f\)の面積分は
            \begin{equation}
                \iint_{\mathcal{D}}fdS=\iint_{\mathcal{D}}f\sqrt{1+\varphi_{x}^{2}+\varphi_{y}^{2}}dxdy
            \end{equation}
        \end{defn}
    \end{leftbar}
    \begin{leftbar}
        \begin{defn}ベクトル場の面積分\par
            \(z=\varphi(x,y)\)で表される曲面における\(\bm{f}\)の面積分は
            \begin{equation}
                \iint_{\mathcal{D}}\bm{f}d\bm{S}=\iint_{\mathcal{D}}\bm{f}\cdot\bm{n}\sqrt{1+\varphi_{x}^{2}+\varphi_{y}^{2}}dxdy
            \end{equation}
        \end{defn}
    \end{leftbar}
    \part{積分定理}
    \section{グリーン[Green]の定理}
    \begin{equation}
        \oint_{\Gamma}f_{1}dx_{1}+f_{2}dx_{2}=\iint_{D}\left(-\frac{\partial f_{1}}{\partial x_{1}}+\frac{\partial f_{2}}{\partial x_{2}}\right )
    \end{equation}
    \section{ガウス[Gauss]の定理}
    \begin{equation}
        \iiint_{V}\mathrm{div}\bm{f}dv=\iint_{A}\bm{f}\cdot d\bm{S}
    \end{equation}
    \section{ストークス[Stokes]の定理}
    \begin{equation}
        \iint_{A}\mathrm{rot}\bm{f}\cdot\bm{n}dS=\oint_{\partial A}\bm{f}\cdot d\bm{r}
    \end{equation}
    \begin{thebibliography}{9}
        \bibitem{text}杉浦光夫. 解析入門I. 初版, 東京大学出版会, 1980, 428p.
    \end{thebibliography}
    \end{document}